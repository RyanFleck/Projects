\documentclass[]{rcf_cv}

%%%%%%%%%%%%%%%%%%%%%%%%%%%%%%%%%%%%%%%%%%%%%%%%%%
%
% Wow! You've found the source code for my resume.
% That's pretty neat. To actually compile this,
% You'll need latex and the texlive-latex-extra
% packages installed on your machine. I'm not
% quite sure how to use LaTeX package managers 
% myself, but if you're reading this, you might
% not either. Cheers!
%
%                                  - Ryan Fleck
%
%%%%%%%%%%%%%%%%%%%%%%%%%%%%%%%%%%%%%%%%%%%%%%%%%%


\begin{document}
	
	\heading{Ryan Fleck}{
		\href{mailto:ryan.fleck@protonmail.com}
			{ryan.fleck@protonmail.com}
		}{
			613 501 4043 --
			\href{https://ryanfleck.ca}{ryanfleck.ca}
		}
	
	\datedsubsection{IBM Associates Program}{Prefered work location is \textbf{Calgary, AB}}{}

	\section{Education}
	
	\datedsubsection{University of Ottawa}{BASc Computer Engineering}{}
		\begin{itemize}
			\setlength\itemsep{-0.4em}
			\renewcommand\labelitemi{--}
			
			\item Graduating Winter 2021, currently in fourth year, eight courses remaining
					
			
			
		\end{itemize}
	
	\section{Experience}
	
		
	\datedsubsection{\href{https://www.ibm.com/}{Wise Assistant}}{Backend Intern}{Summer 2020}
		\begin{itemize}
			\setlength\itemsep{-0.4em}
			\renewcommand\labelitemi{--}
		
			\item Worked in a team of three to develop a Django backend for a new health application
			\item Used Django Rest Framework to write secure model serializers for frontend usage
			\item Adapted multiple third-party authentication libraries to a custom user model
			\item Wrote an image processing pipeline to resize profile pictures and store in Amazon S3
			\item Called the SendGrid API to send account verification and password reset emails

		\end{itemize}
	
	\datedsubsection{\href{https://www.ibm.com/}{IBM}}{Extreme Blue Technical Intern}{Summer 2019}
		\begin{itemize}
			\setlength\itemsep{-0.4em}
			\renewcommand\labelitemi{--}
			
			\item Prototyped a tool to replace deprecated libraries within compiled Java EE binaries
			\item Used Javassist to manipulate bytecode, wrote algorithms to update class functions
			\item Architected a transformation-rule organization system using the reflections library
			\item Created team development workflow with docker, shell \& python scripts, Travis CI

		\end{itemize}
	
	\datedsubsection{\href{https://www.mnp.ca/en}{MNP LLP}}{Co-Op Developer}{January--December 2018}
		\begin{itemize}
			\setlength\itemsep{-0.4em}
			\renewcommand\labelitemi{--}
			
			\item Spoke directly with MNP clients to investigate and address reported bugs
			\item Wrote (in team of three) a replacement Drupal backend for an internal SiteCore CMS
			\item Developed features, fixed bugs in a client's Teamsite (Java EE/Spring) CMS
			\item Wrote technical and user documentation for a client's CMS created with SiteCore 
			\item Wrote AEM HTL/JSP components, configured workflow \& security for bid prototype

			
		\end{itemize}
	
	\section{Volunteering}
	
		\datedsubsection{\href{http://www.uoeracing.com/}{UOE Racing}}{Controller Developer}{September 2019--Present}
		\begin{itemize}
			\setlength\itemsep{-0.4em}
			\renewcommand\labelitemi{--}
			
			\item Responsible for re-writing the embedded brushless DC motor controller code using C
			\item In a team of two, updated code from polling to interrupt-based sensor reading
			
		\end{itemize}
	
	\datedsubsection{\href{https://xalgorithms.org/}{Xalgorithms Foundation}}{Contributor}{April 2018--Present}
		\begin{itemize}
			\setlength\itemsep{-0.4em}
			\renewcommand\labelitemi{--}
			
			\item Implemented a prototype React application to test a potential method of rule writing
			\item Wrote unit tests in Scala for a core system component, the open source rule interpreter
			\item Tested the rule parser against sample rules, proposed improvements, reported bugs

		\end{itemize}

	


	
	\section{Projects}
		\begin{itemize}
			\setlength\itemsep{-0.4em}
			\renewcommand\labelitemi{--}
			
			\item \href{https://github.com/RyanFleck/Influx}{Influx TMS} -- Team Management System written for SEG3102 course with Django
			
			\item Socket.IO Trivia -- Developed a UX trivia webapp in Express with scoreboard \& chat
			
			\item \href{https://github.com/RyanFleck/Fake-News}{Fake News} -- Created a flask app to serve content generated from the given URL
			
			\item \href{https://github.com/RyanFleck/Telegram-Dungeon}{Telegram Chat-Bots} -- Using Python and NodeJS, wrote bots to respond to key words from friends, query Wolfram Alpha, and provide tools for running D\&D games in-chat
			
			\item Games -- Numerous 1-4 player experimental demos in Godot, Unity, Unreal, Phaser
			
			\item \href{https://github.com/morch028/BusBot}{Slack BusBot} -- In a team of two, wrote a slack chat-bot using the OC-Transpo API
			

			
		\end{itemize}

	\newpage

	\section{Technology}
	
		I am familiar with the following technology, and can utilize test-driven design with industry standard design patterns and tools to create and deploy secure business applications.
		
	\textcolor[rgb]{0.99,0.99,0.99}{
	\scalebox{.001}{ % Everything in here is nearly invisible.
	Address: 1205 - 180 Lees Ave Ottawa, Ontario K1S 5J6, 
	City: Ottawa, Province: Ontario ON,
	Name: Ryan Fleck,
	Email: Ryan.Fleck@protonmail.com,
	GitHub: RyanFleck, https://github.com/RyanFleck/
	Stack Overflow: https://stackoverflow.com/users/9899022/ryan-fleck
	Version Control: Git, Github, Microsoft TFS.
	Automated testing with python and selenium.			
	Innovative, Daring, Computer Science, Computer Engineering
	Experience programmer. Computer programmer, 
	Full-stack, 
	Mobile Phone: +1 (613) 501 4043
	Phone: 6135014043
	University: University of Ottawa, 
	Degree Type: Bachelor's Degree, 
	Major: Computer Engineering,	 
	Start: September 2015
	End: April 2021
	Concurrent Java programming.
	Asynchronous JavaScript programming.
	SCRUM, AGILE, technical innovator.			
	} % Make sure there are no line returns before this brace.
	}\\
	\begin{tabular}{l l l l l l l l l}
		Python & Java & Java EE & C\# & .NET & Razor & HTML & CSS & JavaScript\\
		
		C & C++ & Arduino & Ruby & XML & JSON & REST & SOAP & PostgreSQL\\
		
		JUnit & Jest & Android & WS & PHP & Spring & Mongo & Scala &  HC12 ASM\\
		
		Docker & Hugo & Jekyll & Go & XSL & Heroku & SEO & SQL & \LaTeX
		
	\end{tabular} \\
	
	
	\section{Additional Information}

	
	I'd like to elaborate on my \textbf{Java experience} during
	my previous placement at IBM as an Extreme Blue Technical Intern, during other internships, and at the University of Ottawa. \\
	
	The bulk of my experience with Java was tinkering with the Java Virtual Machine and writing
	a sort of re-compiler at IBM. Though I have also used Java extensively while working at MNP
	to build out content management systems, and in class during programming and Software Architecture
	courses, my deepest forays into the JVM took place at IBM, 
	working with the WebSphere and Open Liberty platforms. \\
	
	While I can't divulge the details, I was working in a team with four members; two other technical 
	interns, one business intern, and myself. The goal of the internship was to develop a business pitch
	and working technical prototype before the end of the summer, in order to present to a panel of
	IBM executives in New York. Our application had to take compiled binaries for an older version of
	an enterprise Java system, surgically remove parts of the application that called deprecated functions
	or used outdated libraries, and replace what we could with new/compatible bytecode. The deprecated
	parts of the binary that could not be updated were to be flagged. We estimated that, based on the
	targeted libraries, we would be able to transform at least half of the client applications to
	use modern libraries with zero developer work for the client. As far as I know, the project that
	our team started in 2019 has been carried forward and is currently in development. \\
	
	At MNP, I was tasked to fix bugs in an airline's website/cms. This involved working with a rather
	old copy of Opentext TeamSite, navigating a massive codebase, and learning how to manipulate XML with
	XSLT to present data to the frontend. \\
	
	In my Software Architecture course, we had to develop multiple applications running on GlassFish and
	using Java Server Faces (JSF) as a presentation technology. Though I completed my final project in 
	this course with React/Django, I learned a lot about building flexible, scalable 
	enterprise applications. \\
	
	With all of the above in mind, I've probably had about eight months of real enterprise Java experience.
	This is complemented by years of other programming experience, from the frontend and web applications, 
	all the way down to embedded C. I still have a lot to learn, but I already have a rather solid 
	foundation for new experiences to build upon.
	
	
	
	
\end{document}